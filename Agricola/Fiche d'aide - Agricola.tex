% Appuyez sur la touche [F1] pour compiler ce document

\documentclass[10pt, a4paper]{article}	% Format de page
\usepackage[utf8]{inputenc}				% Pour les caractères accentués
\usepackage[T1]{fontenc}				% Encodage de caractères
\usepackage{lmodern}					% Police vectorielle Latin Modern
\usepackage[frenchb]{babel}				% Règles typographiques françaises
%\usepackage{graphicx}					% Insérer des images
\usepackage{multicol}					% Utiliser plusieurs colonnes
\usepackage{xcolor}						% pour utiliser des couleurs
\usepackage{amssymb}					% symboles mathématiques

\usepackage{geometry}
\geometry{a4paper, landscape, top=0.7cm, bottom=0.5cm, left=1cm, right=1cm}



\begin{document}


{\LARGE \textbf{\textit{Agricola}}}
{\large - Aide de jeu - v1.0}\textit{ -- par Nicolas PENCREACH --}

\setlength{\columnseprule}{0.002cm}	% épaisseur des traits de séparation de colonne
\begin{multicols}{3}				% écrire le contenu sur 3 colonnes


% ####################################
\section{Préparation de la partie}

\begin{itemize}
\item Placer les 3 plateaux d'action au centre. Plateau 1 sur la face familiale pour jouer cette version.
\item Placer le plateau d'aménagements majeurs rempli.
\item Chaque joueur prend ses pions, une ferme et place dessus 2 cabanes en bois et 2 membres de la famille.
\item Mélanger et placer face cachée les cartes actions des différentes périodes.
\item Placer les cartes d'action supplémentaires pour 3 joueurs et +, en respectant la face familiale ou non.
\item Chaque joueur reçoit 7 aménagements mineurs (0 si familial).
\item Prendre les cartes savoir-faire correspondantes au nb de joueurs et en distribuer 7 à chacun (0 si familial).
\item Tirer au sort le 1\ier joueur. Il prend 2PN, les autres 3PN. En solo, pas de PN.
\end{itemize}

\textbf{Important:} seule l'action \og premier joueur\fg{} permet de changer celui-ci.


% ####################################
\section{Description}
Équipé de son lopin de terre, chaque joueur va tenter de faire prospérer au mieux la petite famille qui l'habite en affectant une action à chaque membre par tour.

Une partie est constituée de 14 tours, avec des phases de récolte après certains (4, 7, 9, 11, 13 et 14).



% ####################################
\section{Tour de jeu}

\subsection{Phase I -- Début de tour}
\begin{itemize}
\item Retourner la carte d'action du tour. (S'ajoute aux actions possibles)
\item Récupération de matériel par certains aménagements et savoir-faire joués (Exemple: Puit).
\item Appliquer l'effet de certaines cartes.
\end{itemize}



\subsection{Phase II -- Approvisionnement}
\begin{itemize}
\item Ajouter une marchandise sur les actions avec des flèches. \textbf{Effet cumulatif sans limite}.
\item \textbf{En solo}, ne placer que 2 bois sur la case \og 3 bois\fg{}.
\end{itemize}


\subsection{Phase III -- Travaux}
En commençant par le 1\ier joueur et dans le sens des aiguilles d'une montre, chacun prend \textbf{un} de ses  paysan dans l'habitation et le place sur une case d'action inoccupée, qu'il résout immédiatement.

On procède ainsi jusqu'à ce que tous les paysans soient placés.

\textbf{Règles importantes à toujours respecter :}
\begin{itemize}
\item Un seul paysan par case.
\item L'effet de l'action doit toujours être résolu, sinon il est interdit d'y placer un paysan.
\item \og et/ou\fg{} signifie qu'au moins une des 2 actions doit être réalisée.
\item \og puis\fg{} signifie que la 1\iere{} action est obligatoire, la seconde est \textbf{facultative}.
\item Les matériaux de construction, légumes, céréales et PN sont toujours placés dans les réserves personnelles.
\item Les animaux doivent être hébergés, sinon transformés en PN, sinon replacés dans la réserve générale.
\end{itemize}


\subsection{Phase IV -- Retour à la maison}
Les paysans sont replacés dans la maison, un par pièce.





% ####################################
\section{La récolte}

\subsection{Récolte I -- Les champs}
\begin{itemize}
\item Prendre ds chaque champ cultivé 1 céréale/légume.
\item Certaines cartes donnent aussi des PN à cette phase.
\end{itemize}



\subsection{Récolte II -- Alimentation de la famille}
\begin{itemize}
\item Payer 2 PN par membre de la famille, un seul pour un nouveau né (n'a pas encore fait d'action).
\item Possibilité de convertir des cérales/légumes en 1 PN.
\item Possibilité de cuire des animaux ou d'obtenir de meilleurs taux avec ses bâtiments.
\item \textbf{ATTENTION: il n'est pas possible de cuire du pain à cette étape !!!}
\item Interdit d'abandonner un membre de la famille.
\item Mendier: pour chaque PN manquant, prendre une carte \og mendicité\fg{}. Chaque carte vaut \textbf{-3 pts} en fin de partie !
\end{itemize}


\subsection{Récolte III -- Reproduction du bétail}
Si on possède 2+ animaux de la même espèce, quel que soit leurs emplacements, on obtient un (seul) nouvel animal, \textbf{à condition de pouvoir l'héberger}.

On ne peut pas cuire ce nouvel animal immédiatement, ni ses congénères.





% #######################################
\section{Règles spécifiques aux actions}

\subsection{Actions relatives à l'habitation}

\begin{itemize}
\item Les nouvelles pièces doivent être construites \textbf{à côté} d'une existante (pas en diagonale) et être du \textbf{même type} que le reste de la cabane.
\item La rénovation doit être faite sur toutes les pièces en une fois, pour 1 roseau + X argile/pierre.
\item \textbf{Interdiction de rénover du bois à la pierre.}
\end{itemize}


\subsection{Actions relatives à la famille}
\begin{itemize}
\item 5 membres de famille maximum.
\item Lors de la naissance, on pose le nouveau jeton sur celui qui fait l'action (pas de nouvelle action ce tour) et au \og Retour à la maison\fg{} chacun va dans une pièce différente.
\item 1 pièce d'habitation par membre, sinon pas de naissance.
\end{itemize}


\subsection{Actions relatives aux champs}
\begin{itemize}
\item Un champ peut être construit (labourage) n'importe où la 1\iere{} fois. \`A côté d'un autre ensuite (pas en diagonale).
\item Un champ ne peut être détruit ou déplacé.
\item Semailles : prendre 1 céréale/légume de \textbf{sa} réserve, placer dans un champ et ajouter dessus 2 céréales ou 1 légume pris de la réserve commune.
\end{itemize}



\subsection{Actions relatives au bétail}
On ne peut stocker qu'un animal max. dans la maison. Au delà il faut créer des pâturages, délimités par des clôtures.

\textbf{\\Clôtures}
\begin{itemize}
\item 15 maximum.
\item Doivent délimiter des zones complètement fermées pour être construites.
\item Ne peuvent être détruites ou déplacées.
\end{itemize}

\textbf{\\Pâturages}
\begin{itemize}
\item Les pâturages doivent être adjacents.
\item 1 pâturage par type d'animal.
\item 2 animaux par case de pâturage.
\item Les animaux peuvent être déplacés librement.
\end{itemize}


\textbf{\\Étables}
\begin{itemize}
\item 1 seule par case
\item Ne peut être détruite ou déplacée.
\item Double la capacité du \textbf{pâturage}. Effet cumulable.
\item Peut être construite à l'extérieur d'un pâturage, auquel cas elle permet d'héberger \textbf{1} animal et compte comme une case occupée en fin de partie. Peut également être cloturée par la suite.
\end{itemize}


\subsection{Actions relatives aux cartes}
\begin{itemize}
\item Les aménagements (mineurs et majeurs) peuvent avoir un coût dans le coin sup.droit, des pré-requis dans le coin sup.gauche, un nb de pts de victoire à gauche de l'illustration et des pts variables en fin de partie.
\item Le pré-requis est un minimum.
\item Une fois posée, l'effet de la carte s'applique immédiatement
\end{itemize}



% #######################################
\section{Conseils}
\begin{itemize}
\item \textbf{Anticiper}
	\begin{itemize}
	\item Regarder quelles sont les actions à venir. Naissance en période 2, légume et sanglier à mi-partie, \dots
	\item Pas de naissance ou bétail si pas de place, pas de semances si non labouré, \dots
	\end{itemize}
\item \og Premier joueur\fg{} : donne l'avantage du choix d'action mais coûte une action. \`A jouer avec parcimonie au moment opportun.
\item Monter une usine à nourriture dès les premiers tours pour éviter la mendicité (-3pts).
\item Agrandir la maison et la famille : plus d'actions par tour et 3pts par membre.
\item Éviter de suivre les autres joueurs dans les choix stratégiques, pour éviter les mauvaises surprises.
\item Même si tentant, ne pas jouer trop de cartes savoir-faire et aménagements.
\item Ne pas perdre de vue le décompte final. Éviter de faire une impasse.
\item Maximiser les effets d'une action : beaucoup de clôtures en une fois, \dots
\end{itemize}





% #######################################
\section{Variantes de jeu}
Outre la variation familiale/complète du jeu, il est possible de modifier le système de distribution des cartes savoir-faire et aménagements mineurs, au choix des joueurs en début de partie.

\subsection*{Échange 3-1}
\`A tout moment dans la partie, défausser 3 cartes de sa main (savoir-faire et/ou aménagements mineurs) et piocher face cachée une carte savoir-faire ou aménagement mineur à la place.

\subsection*{7 parmi 10}
Distribuer 10 cartes de chaque et n'en garder que 7, au choix de chacun.

\subsection*{Mulligan}
En début de partie seulement, défausser toutes ses cartes savoir-faire ou aménagement mineur et en piocher 6 nouvelles. On peut recommencer en prenant à chaque fois toutes-1 cartes.

\subsection*{Draft}
Distribuer 7 savoir-faire à chacun. Chaque joueur conserve une carte de côté et passe son paquet au voisin de gauche, puis choisit à nouveau avant de passer son paquet et ainsi de suite.

On procède ensuite de même avec les aménagements.







\end{multicols}
\end{document}
