\documentclass[10pt, a4paper]{article}	% Format de page
\usepackage[utf8]{inputenc}			% Pour les caractères accentués
\usepackage[T1]{fontenc}				% Encodage de caractères
\usepackage{lmodern}				% Police vectorielle Latin Modern
\usepackage[french]{babel}			% Règles typographiques françaises
%\usepackage{graphicx}				% Insérer des images
\usepackage{multicol}				% Utiliser plusieurs colonnes
\usepackage[usenames,dvipsnames]{xcolor} 		% pour utiliser des couleurs https://en.wikibooks.org/wiki/LaTeX/Colors
\usepackage{amssymb}				% symboles mathématiques

\usepackage{geometry}
\geometry{a4paper, landscape, top=0.7cm, bottom=0.5cm, left=1.5cm, right=1cm}

\frenchbsetup{og = «, fg = »}		% remplace automatiquement les guillemets
\frenchbsetup{ItemLabelii=\textbullet}	% redéfini le style des listes de niveau 2

\usepackage{xspace}				% force l’ajout d’espaces


% convertion à la volée de certains caractères utf-8 tapables en bépo vers leur version latex
% (pour s’assurer que les caractères sont bien gérés par latex)
\DeclareUnicodeCharacter{0391}{\ensuremath{\Alpha}}
\DeclareUnicodeCharacter{03B1}{\ensuremath{\alpha}}
\DeclareUnicodeCharacter{0392}{\ensuremath{\Beta}}
\DeclareUnicodeCharacter{03B2}{\ensuremath{\beta}}
\DeclareUnicodeCharacter{0393}{\ensuremath{\Gamma}}
\DeclareUnicodeCharacter{03B3}{\ensuremath{\gamma}}
\DeclareUnicodeCharacter{0394}{\ensuremath{\Delta}}
\DeclareUnicodeCharacter{03B4}{\ensuremath{\delta}}
\DeclareUnicodeCharacter{0395}{\ensuremath{\Epsilon}}
\DeclareUnicodeCharacter{03B5}{\ensuremath{\epsilon}}
\DeclareUnicodeCharacter{0396}{\ensuremath{\Zeta}}
\DeclareUnicodeCharacter{03B6}{\ensuremath{\zeta}}
\DeclareUnicodeCharacter{0397}{\ensuremath{\Eta}}
\DeclareUnicodeCharacter{03B7}{\ensuremath{\eta}}
\DeclareUnicodeCharacter{0398}{\ensuremath{\Theta}}
\DeclareUnicodeCharacter{03B8}{\ensuremath{\theta}}
\DeclareUnicodeCharacter{0399}{\ensuremath{\Iota}}
\DeclareUnicodeCharacter{03B9}{\ensuremath{\iota}}
\DeclareUnicodeCharacter{039A}{\ensuremath{\Kappa}}
\DeclareUnicodeCharacter{03BA}{\ensuremath{\kappa}}
\DeclareUnicodeCharacter{039B}{\ensuremath{\Lambda}}
\DeclareUnicodeCharacter{03BB}{\ensuremath{\lambda}}
\DeclareUnicodeCharacter{039C}{\ensuremath{\Mu}}
\DeclareUnicodeCharacter{03BC}{\ensuremath{\mu}}
\DeclareUnicodeCharacter{039D}{\ensuremath{\Nu}}
\DeclareUnicodeCharacter{03BD}{\ensuremath{\nu}}
\DeclareUnicodeCharacter{039E}{\ensuremath{\Xi}}
\DeclareUnicodeCharacter{03BE}{\ensuremath{\xi}}
\DeclareUnicodeCharacter{039F}{\ensuremath{\Omicron}}
\DeclareUnicodeCharacter{03BF}{\ensuremath{\omicron}}
\DeclareUnicodeCharacter{03A0}{\ensuremath{\Pi}}
\DeclareUnicodeCharacter{03C0}{\ensuremath{\pi}}
\DeclareUnicodeCharacter{03A1}{\ensuremath{\Rho}}
\DeclareUnicodeCharacter{03C1}{\ensuremath{\rho}}
\DeclareUnicodeCharacter{03A3}{\ensuremath{\Sigma}}
\DeclareUnicodeCharacter{03C3}{\ensuremath{\sigma}}
\DeclareUnicodeCharacter{03A4}{\ensuremath{\Tau}}
\DeclareUnicodeCharacter{03C4}{\ensuremath{\tau}}
\DeclareUnicodeCharacter{03A5}{\ensuremath{\Upsilon}}
\DeclareUnicodeCharacter{03C5}{\ensuremath{\upsilon}}
\DeclareUnicodeCharacter{03A6}{\ensuremath{\Phi}}
\DeclareUnicodeCharacter{03C6}{\ensuremath{\varphi}}
\DeclareUnicodeCharacter{03A7}{\ensuremath{\Chi}}
\DeclareUnicodeCharacter{03C7}{\ensuremath{\chi}}
\DeclareUnicodeCharacter{03A8}{\ensuremath{\Psi}}
\DeclareUnicodeCharacter{03C8}{\ensuremath{\psi}}
\DeclareUnicodeCharacter{03A9}{\ensuremath{\Omega}}
\DeclareUnicodeCharacter{03C9}{\ensuremath{\omega}}
\DeclareUnicodeCharacter{00A0}{~}
\DeclareUnicodeCharacter{00AC}{\ensuremath{\neg}}
\DeclareUnicodeCharacter{00B1}{\ensuremath{\pm}}
\DeclareUnicodeCharacter{00D7}{\ensuremath{\times}}
\DeclareUnicodeCharacter{00F7}{\ensuremath{\div}}
\DeclareUnicodeCharacter{2026}{\ldots}
\DeclareUnicodeCharacter{207A}{\ensuremath{^{+}}}
\DeclareUnicodeCharacter{207B}{\ensuremath{^{-}}}
\DeclareUnicodeCharacter{2020}{\ensuremath{\dagger}}
\DeclareUnicodeCharacter{2021}{\ensuremath{\ddagger}}
\DeclareUnicodeCharacter{2212}{\ensuremath{-}}


\begin{document}

% En-tête du document
{\LARGE \textbf{\textit{Lords of Xidit}}}
{\large - Aide de jeu - v1.0}\textit{ -- par Nicolas PENCREACH --}

\setlength{\columnseprule}{0.002cm}	% épaisseur des traits de séparation de colonne
\begin{multicols}{3}				% écrire le contenu sur 3 colonnes


\section{But du Jeu} % (fold)
\label{sec:but_du_jeu}
Chaque joueur incarne un Idrakys (hérault) parcourant le pays pour lever une armée et lutter contre les familiers devenus incontrôlables.
Le joueur ayant le mieux géré sa renommée, sa richesse et son influence restera seul dans les annales comme sauveur du royaume.
% section but_du_jeu (end)


\section{Mise en place} % (fold)
\label{sec:mise_en_place}

\textcolor{BlueViolet}{Pour une mise en place à 3 joueurs, suivre les instructions entre doubles crochets.}
\begin{itemize}
    \item Chaque joueur prend le matériel à sa couleur.
    \item Placer le bastion dans la zone centrale.
    \item Placer les colosses aléatoirement face caché sur leurs emplacements respectifs.
    \item Prendre les tuiles Cité. \textcolor{BlueViolet}{[[À 3 joueurs, celles avec une couronne.]]} Aléatoirement :
    \begin{itemize}
        \item 5 \textcolor{BlueViolet}{[[4]]} face Recrutement (grise) sur les cités correspondantes.
        \item 5 \textcolor{BlueViolet}{[[4]]} face Menace (blanche) sur les cités correspondantes.
        \item 5 \textcolor{BlueViolet}{[[4]]} sur la pile Recrutement.
        \item Le reste (6) sur la pile Menace.
    \end{itemize}
    \item Alimenter la prochaine Menace et prochain Recrutement avec la première tuile de chaque pile.
    \item Placer aléatoirement les 3 tuiles de décompte. À 5 joueurs, la première tuile est marquée de 2 personnages barrés.
    \item Choisir le premier joueur (le plus agé selon la règle).
    \item En commençant par le premier joueur et dans le sens horaire, chacun place son Idrakys sur une cité avec ou sans tuile.\\
            \textbf{Important :} Uniquement durant ce placement initial, 2 joueurs ne peuvent pas occuper la même cité.
    \item \textcolor{BlueViolet}{[[Placer des marqueurs « vide » dans les zones de droite.]]}
    \item \textcolor{BlueViolet}{[[Placer le plateau « joueur neutre » à côté du plateau. Mettre 6, 5 et 4 points sur les colonnes dont le symbole correspond à respectivement à la 1\iere{}, 2\ieme{} et 3\ieme{} tuile Décompte.]]}
\end{itemize}
% section mise_en_place (end)


\section{Déroulement de la partie} % (fold)
\label{sec:deroulement_de_la_partie}
\begin{itemize}
    \item La partie se déroule en 12 années (= tours).
    \item À chaque tour :
    \begin{itemize}
        \item Tous les joueurs programment secrètement leurs 6 ordres.
        \item Les programmations sont révélées.
        \item Le 1er ordre est résolu par chaque joueur, à tour de rôle, en commençant le premier joueur.
        \item Puis idem pour le 2nd ordre et ainsi de suite.
        \item Le marqueur premier joueur passe au suivant dans le sens horaire.
    \end{itemize}
    \item À la fin des tours 4, 8 et 12, avant de changer de premier joueur on effectue un Recensement des Armées.
    \item À la fin du 12ème tour, on procède aux décomptes de points et aux classements.
\end{itemize}
% section déroulement_de_la_partie (end)


\section{Ordres} % (fold)
\label{sec:ordres}
Il existe 3 types d'ordres :
\begin{itemize}
    \item \emph{Déplacement :} l'Idrakys \textbf{doit} suivre le chemin adjacent de la couleur choisie. Sinon sans effet.\\
    Il peut y avoir plusieurs Idrakys sur une même cité.
    \item \emph{Action :} l'Idrakys \textbf{doit} effectuer un recrutement ou éliminer la menace s'il le peut. Sinon sans effet.\\
    Sur une cité vide qui n'a pas eu de \textbf{menace ou colosse} éliminé dans l'\textbf{année} en cours, l'Idrakys \textbf{peux} éliminer un colosse.\\
    \textbf{Important : Un Idrakys ne peux faire qu'une seule action par cité par année.}
    \item \emph{Ne rien faire :} pour passer son tour.
\end{itemize}

\subsection{Recrutement} % (fold)
\label{sub:recrutement}
Le joueur doit prendre la première Unité disponible dans l'ordre des flèches et la place derrière son paravent.

Si la tuile est épuisée, elle est placée dans la défausse et la tuile \og{}prochain recrutement\fg{} entre en jeu sur la cité correspondante et alimentée en unités.\\
La tuile \og{}prochain recrutement\fg{} est alimentée. Si plus de stock, la défausse de menace est retournée et devient la nouvelle pile de recrutement.\\
S'il n'est pas possible d'alimenter le prochain recrutement, on prend la tuile au sommet de la pile menace et on la retourne pour en faire le prochain recrutement.
% subsection recrutement (end)

\subsection{Éliminer une menace} % (fold)
\label{sub:eliminer_une_menace}
Le joueur doit fournir de derrière son paravent le nombre et type d'Unités indiqués sur le bas de la tuile menace et les replacer dans le stock commun.

\subsubsection{Récompense} % (fold)
\label{ssub:recompense}
Il peut choisir ensuite 2 récompenses parmi les 3.
\begin{itemize}
    \item \emph{Étages de Guilde :} Le joueur ajoute autant d'étages à la guilde de magie qu'indiqué.\\
    \textbf{Restrictions importantes :} Une seule guilde par cité, d'une seule couleur. 4 étages maximum.
    \item \emph{Forains d'or :} Le joueur prend autant d'or qu'indiqué et le place derrière son paravent.
    \item \emph{Jetons barde :} Le joueur place à sa convenance autant de jetons barde qu'indiqué dans une ou plusieurs régions adjacentes à son Idrakys. Dans la région centrale les jetons sont placés secrètement dans le fort.\\
    \textbf{Important :} Si le joueur n'a plus de jetons en stock, il ne peux plus recevoir cette récompense.
\end{itemize}
% subsubsection récompense (end)

\subsubsection{Réveil des Colosses} % (fold)
\label{ssub:reveil_des_colosses}
La tuile \og{}prochaine menace\fg{} entre en jeu sur la cité correspondante et remplacé depuis la pile menace.\\
Si la pile est épuisée on retourne la défausse de recrutement qui devient la nouvelle pile de menace.\\
S'il n'est pas possible d'alimenter la prochaine menace, les \textbf{Colosses s'éveillent !}
\begin{itemize}
    \item La première tuile de chaque pile Colosse est retournée face visible.
    \item Retourner la défausse de menace (face recrutement).
    \item Placer dessus la pile de recrutement.
    \item Prendre les 2 premières tuiles de cette pile => nouvelle pile de recrutement.
    \item Mélanger les tuiles restantes, les retourner et toutes les placer sur la pile de menace.
\end{itemize}
% subsubsection réveil_des_colosses (end)
% subsection éliminer_une_menace (end)

\subsection{Éliminer un colosse} % (fold)
\label{sub:eliminer_un_colosse}
Le fonctionnement est simlaire à l'élimination d'une menace à la diférence que le joueur peut utiliser n'importe quel type d'unité.

Le Colosse éliminé est remis dans la boîte, \textbf{le colosse dessous n'est pas retourné}.
% subsection éliminer_un_colosse (end)
% section actions (end)


\section{Recensement des Armées} % (fold)
\label{sec:recensement_des_armees}
Lors de cette phase, les joueurs vont comparer leur unités pour gagner des avantages.
\begin{itemize}
    \item Chacun prend secrêtement dans sa main de derrière son paravent autant de paysan (unité orange) qu'il le souhaite.
    \item Chacun révèle en même temps sa main.
    \item Le joueur ayant révélé le plus d'unités reçoit la récompense.
    \item En cas d'égalité, chacun reçoit une récompense (au lieu de 2 dans certains cas).
    \item Si aucun ne révèle d'unité, aucune récompense n'est attibuée.
    \item L'étage de guilde doit être placé dès que possible sur une guilde et conserver derrière le paravent en attendant.
    \item Pour le placement des jetons barde, les joueurs à égalité les posent en respectant l'ordre de jeu.
    \item Répéter le processus pour chaque type d'unité.
\end{itemize}
% section recensement_des_armées (end)


\section{Décompte des points et Classement} % (fold)
\label{sec:decompte_des_points_et_classement}
À la fin du 12ème tour, après le dernier recensement, pour chacune des 3 colonnes de classement on procède au décompte
des points avec un classement immédiat donnant lieu à une élimination. À la fin des 3 décompte/classement, le joueur
non éliminé le mieux classé remporte la partie.
\begin{itemize}
    \item Richesse : Les joueurs sont classés suivant la quantité d'or possédée.
    \item Influence : Les joueurs sont classés suivant le nombre d'étages de guildes placées sur le plateau.
    Les étages restés derrière le paravent ne comptent pas.
    \item Renommée : Les joueurs marquent des points pour chaque région où ils sont en première ou seconde position en nombre de jeton bardes.\\
    En cas d'égalité en première place, chacun gagne le nombre de points associé et il n'y a pas de deuxième place.\\
    En cas d'égalité pour la seconde place, chacun gagne le nombre de points associé.
\end{itemize}


En cas d'égalité lors du classement, les joueurs sont départagés par leur nb d'unités.\\
Si l'égalité persiste, le joueur le plus loin du premier joueur est le vainqueur de l'égalité.

Le joueur le moins bien classé est éliminé. On retourne son pion et il sera retourné sur les prochains classements.\\
À 5 joueurs, lors du premier classement les 2 derniers sont éliminés.
% section décompte_des_points_et_classement (end)

\section{Variantes} % (fold)
\label{sec:variantes}

\subsection{Version courte en 9 années de jeu} % (fold)
\label{sub:version_courte_9_annees}
Pour une partie plus rapide en 9 années quelques adaptations sont nécessaires.\\
\begin{itemize}
	\item Placer les tuiles Calendrier sur le Calendrier du Plateau de jeu de façon à limiter à 9 années.
	\item Le Recensement des armées à lieu à la fin des 3\ieme{}, 6\ieme{} et 9\ieme{} année de jeu.
	\item À 5 joueurs, retirer les tuiles Cité 20 et 21 et placer un marqueur de vide sur la région 2.
	\item À 4 joueurs, retirer les tuiles Cité 19, 20 et 21 et placer un marqueur de vide sur les régions 1 et 2.
	\item À 3 joueurs, retirer les tuiles Cité 14 et 18 et placer un marqueur de vide sur les régions 1 et 2.
\end{itemize}

\subsection{Pour 3 joueurs} % (fold)
\label{sub:regles_3_joueurs}
À 3 joueurs, la mise en place est légèrement différente (cf. le chapitre mise en place) et introduit un joueur neutre possédant un score initial dans les 3 catégories.\\
À chaque fois qu’un joueur élimine une menace, il doit avancer un des marqueurs de score du joueur neutre.\\
Lors du décompte de points, le joueur neutre peut ainsi éliminer un autre joueur. En cas d’égalité, le neutre est éliminé.

Pour augmenter la difficulté vous pouvez donner 7, 6 et 5 points au joueur neutre au départ.
% section variantes (end)

% ### Fin du document ###
\end{multicols}
\end{document}
