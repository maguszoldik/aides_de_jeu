% Appuyez sur la touche [F1] pour compiler ce document

\documentclass[10pt, a4paper]{article}	% Format de page
\usepackage[utf8]{inputenc}				% Pour les caractères accentués
\usepackage[T1]{fontenc}				% Encodage de caractères
\usepackage{lmodern}					% Police vectorielle Latin Modern
\usepackage[frenchb]{babel}				% Règles typographiques françaises
%\usepackage{graphicx}					% Insérer des images
\usepackage{multicol}					% Utiliser plusieurs colonnes
\usepackage{xcolor}						% pour utiliser des couleurs
\usepackage{amssymb}					% symboles mathématiques

\usepackage{geometry}
\geometry{a4paper, landscape, top=0.7cm, bottom=0.5cm, left=1.5cm, right=1cm}



\begin{document}


{\LARGE \textbf{\textit{Claustrophobia}}}
{\large - Aide de jeu - v1.0}\textit{ -- par Magus Zoldik --}

\setlength{\columnseprule}{0.002cm}	% épaisseur des traits de séparation de colonne
\begin{multicols}{3}				% écrire le contenu sur 3 colonnes

% ####################################
\section{Préparation de la partie}

\begin{itemize}
\item Choisir un scénario (p16 et plus), qui défini la mise en place, condition de victoire et règles spéciales.
\item Les règles spéciales du scénario prédominent toujours sur les règles standard.
\item Les cartes d'avantage de l'humain peuvent être jouées au moment indiqué puis défaussées. Elles ne seront pas repiochées en cours de partie.
\item Les objets sont intimement liés à son possesseur : pas de transmission, perte si mort, \dots
\end{itemize}



% ####################################
\section{Déroulement de la partie}

On joue des tours tant que les conditions de victoire ne sont pas remplies. Un tour se déroule comme suit.

\subsection{Phase d'initiative [joueur Humain]}
\begin{itemize}
\item Lancer un D6 par humain en vie.
\item Associer chaque dé à un humain $\Rightarrow$ indique les caracs du perso jusqu'au prochain tour et les dons actifs pour le frère rédempteur.
\item Une ligne annulée vaut DPT 0, CBT 0, DEF 3.
\item Une ligne annulée empêche l'utilisation des \underline{dons, talents et objets} (sauf \og Sceptre de commandement\fg)
\end{itemize}


\subsection{Phase d'action des Humains}
\begin{itemize}
\item Chaque personnage est activé successivement et complètement.
\item Lors de son activation, un perso peut :
\begin{itemize}
\item \underline{soit} ne rien faire
\item \underline{soit} se déplacer puis combattre
\item \underline{soit} combattre puis se déplacer
\end{itemize}
\end{itemize}



\subsection{Phase de Menace [joueur Démon]}
\begin{itemize}
\item Lancer \textbf{3D6} (ou plus si une action/carte le permet).
\item Répartir les dés dans les cases \og destinés\fg{} à sa convenance, en respectant la condition marquée.
\item Les cases à texte \textcolor{red}{\textbf{rouge}} ne peuvent être sélectionnées qu'une seule fois par tour. Pas de condition sur les \textbf{noires}.
\item Les cartes d'évènements peuvent être conservées sans limite et jouées suivant leur texte, puis défaussées.
\item Les effets sont résolus.
\item Faire entrer des bestioles sur \underline{une ou plusieurs} tuiles :
\begin{itemize}
	\item Coût de 1PM/troglo, 5PM pour un démon (cf. scénario pour ses caracs)
	\item La tuile doit être vide d'humain et comporter au moins une issue non explorée.
	\item Respecter la règle d'encombrement des couloirs.
\end{itemize}
\end{itemize}


\subsection{Phase d'action du démon}
Idem que pour les actions des humains, mais ne peut pas faire d'exploration.





% ####################################
\section{Actions}

\subsection{Déplacement}
\begin{itemize}
\item Le déplacement est facultatif.
\item Aller sur une tuile adjacente coûte 1 DPT.
\item On doit respecter 2 règles :
\begin{enumerate}
	\item \textbf{Règle d'encombrement des couloirs :} pas plus de 3 combattants de son camp/tuile.
	\item \textbf{Règle de blocage :} on peut quitter une tuile \underline{seulement si} nb de combattants de son camp $\geqslant$ nb de combattants du camp adverse.
\end{enumerate}
\end{itemize}


\subsection{Exploration}
Un déplacement peut être fait vers une issue inexplorée (1 DPT). On suit alors cette procédure :
\begin{itemize}
\item L'humain pioche la 1\iere{} tuile et la donne au démon.
\item Le démon la place dans le sens qu'il veut, du moment qu'elle soit accessible.
\item L'humain est placé sur la nouvelle tuile.
\item Les effets de la tuile se déclenchent s'il y a lieu, cf. détail au dos du livret de règle.
\item L'humain peut continuer ses déplacements.
\item \textbf{Cul de sac :} s'il ne reste \underline{aucune} issue non explorée, la dernière tuile est défaussée et on en pioche une nouvelle à la place.
\end{itemize}


\subsection{Combat}
\begin{itemize}
\item L'attaquant indique qui sera la cible.
\item Les troglo comptent pour une seule et même cible.
\item Lance autant de dés que son score de Combat (CBT).
\item \textbf{La DEF de la cible, même avec bonus, ne peut jamais dépasser 6.}
\item Si $Dé\geqslant DEF$ de la cible $\Rightarrow$ 1 touche.
\item Le joueur touché réparti les touches. Pour chacune :
\begin{itemize}
	\item Si troglo $\Rightarrow$ 1 troglo mort (1PV).
	\item Si Démon, 1 marqueur de blessure sur sa carte, jusqu'à atteindre son nombre de PV (puis mort).
	\item Si humain, 1 ligne est annulée. Si elles sont toutes annulées $\Rightarrow$ mort.
\end{itemize}
\end{itemize}



% ####################################
\section{Talents}
Les combattants peuvent avoir des traits ou talents.

\textbf{Insaisissable :}
ignore la \textbf{règle de blocage}.

\textbf{Frénétique :}
relance 1x les dés qui n'ont pas touché.

\textbf{Garde du corps :}
\underline{peut} prendre les touches \underline{d'attaque} faites sur un allié de la même tuile à sa place.

\textbf{Imposant :}
\begin{itemize}
\item Empêche les adversaires de quitter la tuile (sauf sur la tuile \og Trou dans le sol\fg ).
\item Annule le talent \og Insaisissable\fg{}  $\Rightarrow$ La règle normale de blocage s'applique alors.
\end{itemize}

\textbf{Sanctifié :}
\begin{itemize}
\item Utilisable lors de la \underline{phase d'initiative}, \textbf{après le placement des dés}.
\item Désigne un combattant.
\item Celui-ci gagne un bonnus de +1DPT \underline{ou} +2CBT, \underline{jusqu'à la fin de la prochaine phase d'action}.
\item Si la ligne d'action du combattant était annulée, elle est immédiatement guérie.
\item \textbf{Ne peut être utilisé qu'une fois par scénario.}
\end{itemize}


\section{Handicaps \& Experts}
Une section \og Handicap\fg{} (équilibre entre 2 joueurs) et une autre \og Scénarios pour joueurs expérimentés\fg{} (introduit une mise en place différente) sont présentes en p15.

\end{multicols}
\end{document}
