% Appuyez sur la touche [F1] pour compiler ce document

\documentclass[10pt, a4paper]{article}	% Format de page
\usepackage[utf8]{inputenc}				% Pour les caractères accentués
\usepackage[T1]{fontenc}				% Encodage de caractères
\usepackage{lmodern}					% Police vectorielle Latin Modern
\usepackage[frenchb]{babel}				% Règles typographiques françaises
%\usepackage{graphicx}					% Insérer des images
\usepackage{multicol}					% Utiliser plusieurs colonnes
\usepackage{xcolor}						% pour utiliser des couleurs
\usepackage{amssymb}					% symboles mathématiques

\usepackage{geometry}
\geometry{a4paper, landscape, top=0.7cm, bottom=0.7cm, left=1cm, right=1cm}



\begin{document}


{\LARGE \textbf{\textit{Olympos}}}
{\large - Aide de jeu - v1.0}\textit{ -- par Nicolas PENCREACH --}

\setlength{\columnseprule}{0.002cm}	% épaisseur des traits de séparation de colonne
\begin{multicols}{3}				% écrire le contenu sur 3 colonnes

% ####################################
\section{But du jeu}
Les joueurs colonisent la Grèce et ses environs.
En gagnant des territoires et des ressources ils pourront accéder à des découvertes et ainsi construire la plus prestigieuse civilisation.


% ####################################
\section{Préparation}

\begin{itemize}
\item Ne garder que les tuiles découvertes correspondant au nb de joueurs (dos sans numéro et num $\leq$ nb joueurs).
\item Grouper les tuiles identiques (même dessin).
\item Placer aléatoirement les tuiles sur les lignes de leur couleur.
\item Chaque joueur reçoit 4 jetons à sa couleur, représentant son \textbf{stock} de \textbf{colons}.
\item Prendre un 5ème jeton pour chaque joueur et faire une pile sur la première case de la piste de temps (aléatoirement).
\item Le plus en bas de la pile (dernier joueur), prend \textbf{X} jeton(s) \og territoire\fg{} de chaque type et condamne autant de territoires correspondants (même type) en y posant le jeton retourné (croix visible).
	\begin{itemize}
	\item $\textbf{X} = 1/2/3$ pour respectivement $5/4/3$ joueurs.
	\item \textbf{On ne peut pas condamner 2 territoires étoilés d'un même type.}
	\item Condamner des territoires étoilés influe sur la difficulté à bâtir des merveilles.
	\end{itemize}
\item Placer un jeton \og Peuplade\fg{} sur chaque territoire étoilé non condamné.
\item Prendre 4 cubes \og ressource\fg{} différents [+1 aléatoire à 5] et en donner aléatoirement un à chaque joueur.
\item Mélanger les cartes Destin et les cartes Olympos en 2 piles.
\item Retirer de la partie une carte Olympos face cachée.
\end{itemize}



% ####################################
\section{Déroulement de la partie}

\begin{itemize}
\item C'est toujours au dernier joueur sur la piste de temps de jouer.
\item Si 2$+$ joueurs sont en dernière position, c'est au joueur dont le pion est au dessus de jouer.
\item Le joueur réalise une action (Expansion ou Développement).
\item Une fois l'action réalisée, le joueur avance son pion sur l'échelle de temps (le nombre de cases dépend de l'action).
\item Le coût d'une action peut être diminué en utilisant des jetons \og Sablier\fg{}. \textbf{Si le joueur possède des sabliers, il doit les utiliser en priorité avant d'avancer son pion.}
\end{itemize}

\subsection{Action : Expansion}
Cette action consiste à déplacer un colon depuis le plateau ou le stock vers un territoire. Le paiement de toute l'action se fait une fois le déplacement entièrement réalisé.


\vspace{0.1cm}
\textbf{Départ}
\begin{itemize}
\item Nouveau colon : un déplacement depuis le stock coûte 2tps. Le colon arrive alors soit par la zone Nord, soit par un territoire conquis.
\item Quitter un territoire que l'on contrôle fait perdre les jetons correspondants (remis en réserve/sur la case ou passé à un autre joueur occupant la case).
\end{itemize}

\vspace{0.1cm}
\textbf{Déplacement}
\begin{itemize}
\item On peut se déplacer d'autant de cases que souhaité (1 case minimum), en traversant n'importe quelle case (libre, occupée ou condamnée), sauf la zone Nord, sans surcoût.
\item 1 case de terre coûte 1tps.
\item 1 case de mer coûte 2tps.
\end{itemize}

\vspace{0.1cm}
\textbf{Arrivée}
\begin{itemize}
\item \textbf{On ne peut pas s'arrêter dans un territoire condamné (croix).}
\item Quand on s'arrête sur un territoire, on en prend automatiquement le contrôle.
\item Si on s'arrête sur le territoire Olympos ou une étoilé, on prend le jeton.
\item Si le territoire était vide, on prend un jeton correspondant dans la réserve.
\item Si le territoire est occupé par un joueur :
	\begin{itemize}
	\item Il y a un combat, remporté automatiquement.
	\item \textbf{Coûte 1/2/3 tps si respectivement $+$/$=$/$-$ d'épées.}
	\item Le pion arrivant est placé sur celui déjà présent.
	\item Le contrôle passe au nouvel arrivant, qui reçoit le jeton territoire du \og vaincu\fg{} et l'éventuel jeton Zeus ou étoile de la case.
	\item \textbf{Le vaincu prend un jeton sablier dans la réserve.}
	\item Le vaincu peut rester sur le territoire tant qu'il veut mais devra le quitter puis revenir et combattre s'il veut en reprendre le contrôle, à moins que l'autre joueur ne le quitte de lui même.\textbf{ Tant qu'il ne contrôle pas le territoire il ne peut pas y faire venir de nouveaux colons.}
	\end{itemize}
\item Si le territoire est occupé par une Peuplade, il y a un combat comme avec un joueur et donc un coût en tps (Peuplade = 0 épée).
\textbf{La Peuplade est transformée en étoile et le restera jusqu'à la fin de la partie.}
\end{itemize}

\vspace{0.3cm}
\textbf{Coût de l'action $=$ nouveau colon $+$ déplacement $+$ combat.}



\subsection{Action : Développement}
Cette action consiste à acheter une tuile de la plaque de développements, contre \textbf{7tps} à la fin de l'action.

\vspace{0.1cm}
\textbf{Découvertes}
\begin{itemize}
\item Payables en territoire et/ou cube (ressource).
\item Les territoires restent devant le joueur.
\item Les cubes sont défaussés.
\item Les \og $+$\fg{} dans le prix indiquent que les X ressources doivent être \textbf{d'un type} différent des précédents.
\item Le joueur prend la tuile (au choix si plusieurs).
\item Le joueur prend un jeton \textbf{de la réserve} qu'il place sur le bonus de son choix, immédiatement résolu. \textbf{Le fond gris est réservé pour les parties à 5.}
\item \textbf{Il est interdit de prendre 2 fois la même découverte.}
\end{itemize}


\vspace{0.1cm}
\textbf{Merveilles}\\
Le joueur doit disposer du nombre d'étoiles indiqué dessus. Ces étoiles sont
procurées par
\begin{itemize}
\item Les jetons étoile gagnés lors de la prise d'un territoire.
\item Certaines cartes Destin et Olympos.
\item Les découvertes Architecture et Ingénierie.
\item Chaque jeton bonus situé \textbf{dans la même colonne} que la merveille.
\end{itemize}


\subsection{Cases Zeus}
\textbf{Cartes Destin}\\
Dès qu'un joueur atteint ou franchi une case Zeus, il pioche une carte Destin qu'il peut jouer immédiatement ou à n'importe quel moment de son tour.

\vspace{0.1cm}
\textbf{Carte Olympos}
\begin{itemize}
\item Le \textbf{premier joueur} à atteindre une case Zeus déclenche le tirage d'une carte Olympos après avoir reçu sa carte de Destin.
\item Si un deuxième symbole Zeus est présent sur la case, une seconde carte est tirée quand le \textbf{dernier joueur} atteint à son tour la case.
\item La moitié des effets est positif, l'autre négatif.
\item Les effets positifs touchent ceux ayant le plus de points Zeus (\textbf{1 minimum}).
\item Les effets négatifs touchent les joueurs en ayant le moins.
\end{itemize}


\subsection{Fin du jeu}
Une fois qu'un joueur dépasse la dernière case Zeus, il peut décider de soit \textbf{passer}, soit \textbf{effectuer une dernière action},
sans dépasser la croix en coût de tps.\\
Une fois celà fait, il a fini de jouer.\\
La partie prend fin quand tous les joueurs se retrouvent dans cette situation.


% ####################################
\section{Décompte des points}
\begin{itemize}
\item Points de la piste de temps
\item Points des jetons Prestige
\item 1pt/territoire normal sous contrôle
\item \textbf{2pts/territoire Atlantide sous contrôle}
\item 2pts/tuile découverte
\item Points des Découvertes
\item Points des Merveilles
\item La carte \og Kères\fg{} fait perdre 2pts à ses victimes
\item \textbf{1pt/carte Destin encore en main}
\end{itemize}

En cas d'égalité, le max de tuiles découverte + merveille l'emporte.



\end{multicols}
\end{document}
