% !TeX encoding = UTF-8

\documentclass[10pt, a4paper]{article}	% Format de page
\usepackage[utf8]{inputenc}				% Pour les caractères accentués
\usepackage[T1]{fontenc}				% Encodage de caractères
\usepackage{lmodern}					% Police vectorielle Latin Modern
\usepackage[french]{babel}				% Règles typographiques françaises
%\usepackage{graphicx}					% Insérer des images
\usepackage{multicol}					% Utiliser plusieurs colonnes
\usepackage[usenames,dvipsnames]{xcolor} % pour utiliser des couleurs https://en.wikibooks.org/wiki/LaTeX/Colors
\usepackage{amssymb}					% symboles mathématiques

\usepackage{geometry}
\geometry{a4paper, landscape, top=0.7cm, bottom=5.6cm, left=1.25cm, right=1.25cm}

\pagestyle{empty}			% supprimer les numéros de pages

\frenchbsetup{og = «, fg = »}		% remplace automatiquement les guillemets
\frenchbsetup{ItemLabelii=\textbullet}	% redéfini le style des listes de niveau 2

\usepackage{xspace}				% force l’ajout d’espaces

% convertion à la volée de certains caractères utf-8 tapables en bépo vers leur version latex
% (pour s’assurer que les caractères sont bien gérés par latex)
\DeclareUnicodeCharacter{0391}{\ensuremath{\Alpha}}
\DeclareUnicodeCharacter{03B1}{\ensuremath{\alpha}}
\DeclareUnicodeCharacter{0392}{\ensuremath{\Beta}}
\DeclareUnicodeCharacter{03B2}{\ensuremath{\beta}}
\DeclareUnicodeCharacter{0393}{\ensuremath{\Gamma}}
\DeclareUnicodeCharacter{03B3}{\ensuremath{\gamma}}
\DeclareUnicodeCharacter{0394}{\ensuremath{\Delta}}
\DeclareUnicodeCharacter{03B4}{\ensuremath{\delta}}
\DeclareUnicodeCharacter{0395}{\ensuremath{\Epsilon}}
\DeclareUnicodeCharacter{03B5}{\ensuremath{\epsilon}}
\DeclareUnicodeCharacter{0396}{\ensuremath{\Zeta}}
\DeclareUnicodeCharacter{03B6}{\ensuremath{\zeta}}
\DeclareUnicodeCharacter{0397}{\ensuremath{\Eta}}
\DeclareUnicodeCharacter{03B7}{\ensuremath{\eta}}
\DeclareUnicodeCharacter{0398}{\ensuremath{\Theta}}
\DeclareUnicodeCharacter{03B8}{\ensuremath{\theta}}
\DeclareUnicodeCharacter{0399}{\ensuremath{\Iota}}
\DeclareUnicodeCharacter{03B9}{\ensuremath{\iota}}
\DeclareUnicodeCharacter{039A}{\ensuremath{\Kappa}}
\DeclareUnicodeCharacter{03BA}{\ensuremath{\kappa}}
\DeclareUnicodeCharacter{039B}{\ensuremath{\Lambda}}
\DeclareUnicodeCharacter{03BB}{\ensuremath{\lambda}}
\DeclareUnicodeCharacter{039C}{\ensuremath{\Mu}}
\DeclareUnicodeCharacter{03BC}{\ensuremath{\mu}}
\DeclareUnicodeCharacter{039D}{\ensuremath{\Nu}}
\DeclareUnicodeCharacter{03BD}{\ensuremath{\nu}}
\DeclareUnicodeCharacter{039E}{\ensuremath{\Xi}}
\DeclareUnicodeCharacter{03BE}{\ensuremath{\xi}}
\DeclareUnicodeCharacter{039F}{\ensuremath{\Omicron}}
\DeclareUnicodeCharacter{03BF}{\ensuremath{\omicron}}
\DeclareUnicodeCharacter{03A0}{\ensuremath{\Pi}}
\DeclareUnicodeCharacter{03C0}{\ensuremath{\pi}}
\DeclareUnicodeCharacter{03A1}{\ensuremath{\Rho}}
\DeclareUnicodeCharacter{03C1}{\ensuremath{\rho}}
\DeclareUnicodeCharacter{03A3}{\ensuremath{\Sigma}}
\DeclareUnicodeCharacter{03C3}{\ensuremath{\sigma}}
\DeclareUnicodeCharacter{03A4}{\ensuremath{\Tau}}
\DeclareUnicodeCharacter{03C4}{\ensuremath{\tau}}
\DeclareUnicodeCharacter{03A5}{\ensuremath{\Upsilon}}
\DeclareUnicodeCharacter{03C5}{\ensuremath{\upsilon}}
\DeclareUnicodeCharacter{03A6}{\ensuremath{\Phi}}
\DeclareUnicodeCharacter{03C6}{\ensuremath{\varphi}}
\DeclareUnicodeCharacter{03A7}{\ensuremath{\Chi}}
\DeclareUnicodeCharacter{03C7}{\ensuremath{\chi}}
\DeclareUnicodeCharacter{03A8}{\ensuremath{\Psi}}
\DeclareUnicodeCharacter{03C8}{\ensuremath{\psi}}
\DeclareUnicodeCharacter{03A9}{\ensuremath{\Omega}}
\DeclareUnicodeCharacter{03C9}{\ensuremath{\omega}}
\DeclareUnicodeCharacter{00A0}{~}
\DeclareUnicodeCharacter{00AC}{\ensuremath{\neg}}
\DeclareUnicodeCharacter{00B1}{\ensuremath{\pm}}
\DeclareUnicodeCharacter{00D7}{\ensuremath{\times}}
\DeclareUnicodeCharacter{00F7}{\ensuremath{\div}}
\DeclareUnicodeCharacter{2026}{\ldots}
\DeclareUnicodeCharacter{207A}{\ensuremath{^{+}}}
\DeclareUnicodeCharacter{207B}{\ensuremath{^{-}}}
\DeclareUnicodeCharacter{2020}{\ensuremath{\dagger}}
\DeclareUnicodeCharacter{2021}{\ensuremath{\ddagger}}
\DeclareUnicodeCharacter{2212}{\ensuremath{-}}
\DeclareUnicodeCharacter{2260}{\ensuremath{\neq}}


\begin{document}

% En-tête du document
{\LARGE \textbf{\textit{Gosu}}}
{\large - Aide de jeu - v2.1}\textit{ — par Nicolas PENCREACH —}

\setlength{\columnseprule}{0.005cm}	% épaisseur des traits de séparation de colonne
\begin{multicols}{3}				% écrire le contenu sur 3 colonnes

\textcolor{Plum}{[Ajout/modification lié à l’extension Kamakor.]}

% #################################
\section{Présentation}
\label{sec:presentation}

\begin{itemize}
	\item 100 cartes de gobelins
	\item 5 clans (= couleurs/symboles)
	\item 3 types/niveaux dans chaque :
	\begin{itemize}
		\item I   — 50 Bakutos (25 doubles)
		\item II  — 35 Héros (uniques)
		\item III — 15 Ozekis (uniques)
	\end{itemize}
\end{itemize}


But : construire la plus grande armée et accumuler 3 points de victoire \textbf{OU} remplir une condition de victoire alternative d’un Ozeki.


% #################################
\section{Préparation}
\label{sec:preparation}

\begin{itemize}
	\item Définir au hasard qui reçoit le jeton avantage.
	\item \textcolor{Plum}{Chaque joueur élimine un des 10 clans et ainsi de suite pour n’en conserver que 5.}
	\item Mélanger les cartes en une pile.
	\item Distribuer 2 jetons d’activation à chacun.
	\item Chacun pioche 7 cartes. Si pas de niveau I, on peut défausser toute sa main et repiocher (et ainsi de suite) jusqu’à en avoir au moins un.
\end{itemize}


% #################################
\section{Fonctionnement}
\label{sec:fonctionnement}

\begin{itemize}
	\item Le jeu se déroule en rondes.
	\item Une ronde se découpe en tours et se termine quand tous les joueurs ont passé.
	\item À chaque tour, un joueur peut soit faire une action soit passer, auquel cas il ne joue plus avant la fin de la ronde.
	\\Note : le dernier joueur dans la ronde peut continuer seul tant qu'il le souhaite.
	\item À la fin de la ronde il y a une Grande Bataille. On compte les points de chaque armée et la plus forte gagne un point de victoire.
	\\En cas d'égalité celui avec le jeton avantage gagne le point si présent, sinon tous gagnent un point.
	\item On recommence ensuite une ronde :
	\begin{itemize}
		\item On récupère ses jetons d'activation.
		\item Libérer (= retourner) les cartes emprisonnées.
		\item \textbf{Le jeton avantage reste où il est}.
		\item \textbf{On ne repioche pas de cartes}.
		\item \textcolor{Plum}{Les jetons Dragoon joués sur des cartes sont remis dans leur réserve.}
	\end{itemize}
\end{itemize}

\vspace{0.2cm}
Les actions possibles sont :
\begin{itemize}
	\item Poser un gobelin (voir plus loin).
	\item Muter un gobelin (voir plus loin).
	\item Dépenser un jeton activ. pour piocher une carte.
	\item Dépenser 2 jetons activ. pour piocher 3 cartes.
	\item Activer 1 pouvoir avec 1 jeton activ. \textcolor{Plum}{(ou Dragoon)}.
	\\ \textbf{1 jeton max. par gobelin}.
	\item Passer son tour. \textcolor{Plum}{Un joueur ne peut pas passer tant qu’il lui reste un jeton d’activation.}
\end{itemize}

% -----------------------------
\subsection{Poser un gobelin}
\begin{itemize}
	\item Les gobelins sont posés depuis la main.
	\item Ils sont posés en ligne suivant leurs niveaux (donc 3 lignes max), de la gauche vers la droite.
	\item 5 par ligne max.
	\item Au niveau I :
	\begin{itemize}
		\item le premier est toujours gratuit.
		\item les suivants coûtent 2 cartes à défausser si c'est une nouvelle couleur, gratuit sinon.
	\end{itemize}
	\item Au niveau II :
	\begin{itemize}
		\item gratuit mais on ne peux poser qu’une couleur déjà présente au niveau I.
		\item pas plus de carte qu'au niveau I (pyramide).
	\end{itemize}
	\item Au niveau III :
	\begin{itemize}
		\item gratuit mais on ne peux poser qu'une couleur déjà présente au niveau II \textbf{ET} au niveau I.
		\item pas plus de cartes qu’au niveau II (pyramide).
	\end{itemize}
\end{itemize}

% ----------------------------------
\subsection{Muter un gobelin}
\begin{itemize}
	\item S'il y a un nombre à coté du yin-yang alors la carte peut être mutée (= remplacée).
	\item Le nombre indique la quantité de cartes à défausser pour la muter.
	\item On peut muter une carte en n'importe quelle autre de même niveau présente dans sa main, sauf carte identique.
	\item Il est possible de muter dans une couleur non présente dans les niveaux inférieurs.
	\item La \textbf{mutation zombie} permet de muter en un gobelin présent dans la défausse.
	\item \color{Plum} Le \textbf{Shadow Jump} est une forme de mutation particulière. Au lieu de payer un coup en défausse, on retourne la première carte de la pioche : 
	\begin{itemize}
		\item Si même niveau $\Rightarrow$ remplacée.
		\item Sinon la carte est emprisonnée.
	\end{itemize}
\end{itemize}


% #################################
\section{Points particuliers}
\label{sec:points_particuliers}
\begin{itemize}
	\item Les Bakutos (niveau I) donnent le nom du clan.
	\item Il n'y a pas de limite de carte en main.
	\item La défausse peut être consultée à n'importe quel moment.
	\item La défausse est remélangée à la pioche quand on doit piocher et qu'il n'y a plus de carte pour ça.
	\item Sauf mention explicite, les effets peuvent cibler les cartes personnelles et adverses
	\item Une carte est dite \textbf{LIBRE} si elle n'a pas de carte à droite ou au-dessus d'elle. \\ \textbf{Elles sont les seules qu'on peut détruire.}
	\item Certaines cartes ont un bonus entre parenthèses qui s'applique \textit{uniquement} si un adversaire a plus de points de victoire.
	\item Les Ozekis (niveau III) ont des pouvoirs qui s'appliquent à tous les joueurs, peuvent changer les règles et ajouter des conditions de victoire ; précédé de la mention « global ».
	\item Une carte emprisonnée est retournée : elle n'a alors plus de valeur, pouvoir, ni clan. Elle peut être ciblée comme « libre » et on peut poser au dessus.
	\item \textcolor{Plum}{L’effet de Kao, Sombre Gobelin de niveau II, ne se déclenche pas si une carte s’emprisonne elle même.}
	\item \textcolor{Plum}{Certaines cartes peuvent en « libérer » d’autres (retourner face visible).}
	\item Si plusieurs effets arrivent en même temps, le joueur actif choisit l'ordre de résolution mais tous les effets doivent être appliqués et aucun ne peut être annulé à cause de l'ordre de résolution.
	\item Un joueur gagne dès qu'il obtient 3 PV et met fin à la partie immédiatement.
	\item \textcolor{Plum}{Une carte est dite \textbf{première carte} quand elle est placée tout en bas à gauche.}
	\item \textcolor{Plum}{Une carte \textbf{RECRUTEUR} indique que la carte posée juste à droite (≠ mutée) sera gratuite quelle que soit sa couleur.\\ \textbf{Quand elle est détruite, tous les adversaires piochent une carte.}}
	\item \textcolor{Plum}{2 jetons Dragoon max. en main + 2 posés max.}
	\item \textcolor{Plum}{S’il n’y a plus de jeton Dragoon en réserve, on ne peut plus en gagner.}
	\item \textcolor{Plum}{Certaines cartes peuvent être activées en remplissant l’action inscrite en italique précédée de « : ». Cette activation compte comme une action mais peut-être réalisée plusieurs fois par ronde.}
\end{itemize}


\vspace{4cm}
% #################################
\section{Règles optionnelles}
\label{sec:regles_optionnelles}

Ces règles sont issues des règles de tournoi et présentées dans l’extension Kamakor. Elle peuvent toutefois parfaitement être employées avec le jeu de base.

\subsection*{Draft}
Une fois les mains de départ reçues, chaque joueur garde 2 cartes et fait passer sa main au voisin de gauche. Puis chacun conserve une carte et fait à nouveau passer son jeu à gauche et ainsi de suite jusqu’à épuisement.


\subsection*{Limitation des tours}
Quand un premier joueur passe, les autres ne peuvent alors plus jouer qu’un maximum de 10 fois \textbf{en tout (≠ chacun)}.\\
Si un effet de carte donne un tour supplémentaire, ce tour bonus est compté dans les 10.
Si lors du 10\ieme tour un effet donne un tour supplémentaire, ce tour bonus (et éventuellement les suivants) est joué.


\subsection*{Jeu en équipe (2 VS 2)}
\begin{itemize}
	\item La victoire se joue en 3 pts \textbf{pour l’équipe}.
	\item La cible « un joueur » peut désigner son allié.
	\item La cible « un adversaire » désigne obligatoirement quelqu’un de l’autre camp.
	\item On peut jouer ses jetons (activation \textcolor{Plum}{et dragoon}) pour son allié afin de le faire piocher ou activer des effets de ses cartes qui seront joués comme s’il avait lui même utilisé un jeton.
	\item Pour la grande bataille, seule la plus puissante des armées de chaque camp est prise en compte. En cas d’égalité on compare les 2 autres. Si toujours égalité, l’équipe avec l’avantage remporte le point.
	\item Pour la limitation de tours, la règle des 10 coups s’applique dès que les 2 membres d’une même équipe ont passé.
\end{itemize}




% ### Fin du document ###
\end{multicols}
\end{document}
