\documentclass[10pt, a4paper]{article}	% Format de page
\usepackage[utf8]{inputenc}			% Pour les caractères accentués
\usepackage[T1]{fontenc}				% Encodage de caractères
\usepackage{lmodern}				% Police vectorielle Latin Modern
\usepackage[french]{babel}			% Règles typographiques françaises
%\usepackage{graphicx}				% Insérer des images
\usepackage{multicol}				% Utiliser plusieurs colonnes
\usepackage{xcolor}					% pour utiliser des couleurs
\usepackage{amssymb}				% symboles mathématiques

\usepackage{geometry}
\geometry{a4paper, landscape, top=0.7cm, bottom=0.5cm, left=1.5cm, right=1cm}

\frenchbsetup{og = «, fg = »}		% remplace automatiquement les guillemets
\frenchbsetup{ItemLabelii=\textbullet}	% redéfini le style des listes de niveau 2

\usepackage{xspace}				% force l’ajout d’espaces

% convertion à la volée de certains caractères utf-8 tapables en bépo vers leur version latex
% (pour s’assurer que les caractères sont bien gérés par latex)
\DeclareUnicodeCharacter{0391}{\ensuremath{\Alpha}}
\DeclareUnicodeCharacter{03B1}{\ensuremath{\alpha}}
\DeclareUnicodeCharacter{0392}{\ensuremath{\Beta}}
\DeclareUnicodeCharacter{03B2}{\ensuremath{\beta}}
\DeclareUnicodeCharacter{0393}{\ensuremath{\Gamma}}
\DeclareUnicodeCharacter{03B3}{\ensuremath{\gamma}}
\DeclareUnicodeCharacter{0394}{\ensuremath{\Delta}}
\DeclareUnicodeCharacter{03B4}{\ensuremath{\delta}}
\DeclareUnicodeCharacter{0395}{\ensuremath{\Epsilon}}
\DeclareUnicodeCharacter{03B5}{\ensuremath{\epsilon}}
\DeclareUnicodeCharacter{0396}{\ensuremath{\Zeta}}
\DeclareUnicodeCharacter{03B6}{\ensuremath{\zeta}}
\DeclareUnicodeCharacter{0397}{\ensuremath{\Eta}}
\DeclareUnicodeCharacter{03B7}{\ensuremath{\eta}}
\DeclareUnicodeCharacter{0398}{\ensuremath{\Theta}}
\DeclareUnicodeCharacter{03B8}{\ensuremath{\theta}}
\DeclareUnicodeCharacter{0399}{\ensuremath{\Iota}}
\DeclareUnicodeCharacter{03B9}{\ensuremath{\iota}}
\DeclareUnicodeCharacter{039A}{\ensuremath{\Kappa}}
\DeclareUnicodeCharacter{03BA}{\ensuremath{\kappa}}
\DeclareUnicodeCharacter{039B}{\ensuremath{\Lambda}}
\DeclareUnicodeCharacter{03BB}{\ensuremath{\lambda}}
\DeclareUnicodeCharacter{039C}{\ensuremath{\Mu}}
\DeclareUnicodeCharacter{03BC}{\ensuremath{\mu}}
\DeclareUnicodeCharacter{039D}{\ensuremath{\Nu}}
\DeclareUnicodeCharacter{03BD}{\ensuremath{\nu}}
\DeclareUnicodeCharacter{039E}{\ensuremath{\Xi}}
\DeclareUnicodeCharacter{03BE}{\ensuremath{\xi}}
\DeclareUnicodeCharacter{039F}{\ensuremath{\Omicron}}
\DeclareUnicodeCharacter{03BF}{\ensuremath{\omicron}}
\DeclareUnicodeCharacter{03A0}{\ensuremath{\Pi}}
\DeclareUnicodeCharacter{03C0}{\ensuremath{\pi}}
\DeclareUnicodeCharacter{03A1}{\ensuremath{\Rho}}
\DeclareUnicodeCharacter{03C1}{\ensuremath{\rho}}
\DeclareUnicodeCharacter{03A3}{\ensuremath{\Sigma}}
\DeclareUnicodeCharacter{03C3}{\ensuremath{\sigma}}
\DeclareUnicodeCharacter{03A4}{\ensuremath{\Tau}}
\DeclareUnicodeCharacter{03C4}{\ensuremath{\tau}}
\DeclareUnicodeCharacter{03A5}{\ensuremath{\Upsilon}}
\DeclareUnicodeCharacter{03C5}{\ensuremath{\upsilon}}
\DeclareUnicodeCharacter{03A6}{\ensuremath{\Phi}}
\DeclareUnicodeCharacter{03C6}{\ensuremath{\varphi}}
\DeclareUnicodeCharacter{03A7}{\ensuremath{\Chi}}
\DeclareUnicodeCharacter{03C7}{\ensuremath{\chi}}
\DeclareUnicodeCharacter{03A8}{\ensuremath{\Psi}}
\DeclareUnicodeCharacter{03C8}{\ensuremath{\psi}}
\DeclareUnicodeCharacter{03A9}{\ensuremath{\Omega}}
\DeclareUnicodeCharacter{03C9}{\ensuremath{\omega}}
\DeclareUnicodeCharacter{00A0}{~}
\DeclareUnicodeCharacter{00AC}{\ensuremath{\neg}}
\DeclareUnicodeCharacter{00B1}{\ensuremath{\pm}}
\DeclareUnicodeCharacter{00D7}{\ensuremath{\times}}
\DeclareUnicodeCharacter{00F7}{\ensuremath{\div}}
\DeclareUnicodeCharacter{2026}{\ldots}
\DeclareUnicodeCharacter{207A}{\ensuremath{^{+}}}
\DeclareUnicodeCharacter{207B}{\ensuremath{^{-}}}
\DeclareUnicodeCharacter{2020}{\ensuremath{\dagger}}
\DeclareUnicodeCharacter{2021}{\ensuremath{\ddagger}}
\DeclareUnicodeCharacter{2212}{\ensuremath{-}}


\begin{document}

% En-tête du document
{\LARGE \textbf{\textit{Gosu}}}
{\large - Aide de jeu - v1.0}\textit{ — par Nicolas PENCREACH —}

\setlength{\columnseprule}{0.002cm}	% épaisseur des traits de séparation de colonne
\begin{multicols}{3}				% écrire le contenu sur 3 colonnes

% #################################
\section{Présentation}
\label{sec:presentation}

\begin{itemize}
\item 100 cartes de gobelins
\item 5 couleurs
\item 3 niveaux (types) :
	\begin{itemize}
	\item I   — 50 Bakutos (25 doubles)
	\item II  — 35 Héros (uniques)
	\item III — 15 Ozekis (uniques)
	\end{itemize}
\end{itemize}

But : construire la plus grande armée et accumuler 3 points de victoire \textbf{OU} remplir une condition de victoire alternative d’un Ozeki


% #################################
\section{Préparation}
\label{sec:preparation}

\begin{itemize}
\item Mélanger les cartes en une pile.
\item Distribuer 2 jetons d’activation à chacun.
\item Définir au hasard qui reçoit le jeton avantage.
\item Chacun pioche 7 cartes. Si pas de niveau I, peut défausser sa main et repiocher (et ainsi de suite) jusqu’à en avoir au moins un.
\end{itemize}


% #################################
\section{Fonctionnement}
\label{sec:fonctionnement}

\begin{itemize}
\item Le jeu se déroule en rondes.
\item Une ronde se découpe en tours et se termine quand tous les joueurs ont passé.
\item À chaque tour, un joueur peut soit faire une action soit passer, auquel cas il ne joue plus avant la fin de la ronde.
\\Note : le dernier joueur dans la ronde peut continuer seul tant qu'il le souhaite.
\item À la fin de la ronde il y a une Grande Bataille. On compte les points de chaque armée et la plus forte gagne un point de victoire.
\\En cas d'égalité celui avec le jeton avantage gagne le point si présent, sinon tous gagnent un point.
\item On recommence ensuite une ronde :
	\begin{itemize}
	\item on récupère ses jetons d'activation.
	\item les cartes emprisonnées sont retournées.
	\item \textbf{le jeton avantage reste où il est}.
	\item \textbf{on ne repioche pas de cartes}.
	\end{itemize}
\end{itemize}

\vspace{0.2cm}
Les actions possibles sont :
\begin{itemize}
\item poser un gobelin (voir plus loin).
\item muter un gobelin (voir plus loin).
\item dépenser un jeton pour piocher une carte.
\item dépenser 2 jetons pour piocher 3 cartes.
\item poser 1 jeton sur 1 gobelin pour activer son pouvoir.
\item (passer son tour.)
\end{itemize}

% -----------------------------
\subsection{Poser un gobelin}
\begin{itemize}
\item Les gobelins sont posés depuis la main.
\item Ils sont posés en ligne suivant leurs niveaux (donc 3 lignes max), de la gauche vers la droite.
\item 5 par ligne max.
\item Au niveau I :
	\begin{itemize}
	\item le premier est toujours gratuit.
	\item les suivants coûtent 2 cartes à défausser si c'est une nouvelle couleur, gratuit sinon.
	\end{itemize}
\item Au niveau II :
	\begin{itemize}
	\item gratuit mais on ne peux poser qu’une couleur déjà présente au niveau I.
	\item pas plus de carte qu'au niveau I (pyramide).
	\end{itemize}
\item Au niveau III :
	\begin{itemize}
	\item gratuit mais on ne peux poser qu'une couleur déjà présente au niveau II ET au nideau I.
	\item pas plus de cartes qu’au niveau II (pyramide).
	\end{itemize}
\end{itemize}

% ----------------------------------
\subsection{Muter un gobelin}
\begin{itemize}
\item S'il y a un nombre à coté du yin-yang alors la carte peut être mutée (= remplacée).
\item Le nombre indique la quantité de cartes à défausser pour la muter.
\item On peut muter une carte en n'importe quelle autre de même niveau présente dans sa main, sauf carte identique.
\item Il est possible de muter dans une couleur non présente dans les niveaux inférieurs.
\item La mutation zombie, spécifique aux gobelins noirs, permet de muter en un gobelin présent dans la défausse.
\end{itemize}


% #################################
\section{Points particuliers}
\label{sec:points_particuliers}
\begin{itemize}
\item Il n'y a pas de limite de carte en main.
\item La défausse peut être consultée à n'importe quel moment.
\item La défausse est remélangée à la pioche quand quelqu'un doit piocher et qu'il n'y a plus de carte pour ça.
\item Sauf mention explicite, les effets peuvent cibler les cartes personnelles et adverses.
\item Une carte est dite \textbf{LIBRE} si elle n'a pas de carte à droite ou au-dessus d'elle. \\ \textbf{Elles sont les seules qu'on peut détruire.}
\item Certaines cartes ont un bonus entre parenthèses qui s'applique \textit{uniquement} si un adversaire a plus de points de victoire.
\item Les Ozekis (niveau III) ont des pouvoirs qui s'appliquent à tous les joueurs, peuvent changer les règles et ajouter des conditions de victoire ; précédé de la mention « global ».
\item Une carte emprisonnée est retournée : elle n'a alors plus de valeur, pouvoir, ni clan. Elle peut être ciblée comme « libre » et on peut poser au dessus.
\item Si plusieurs effets arrivent en même temps, le joueur actif choisit l'ordre de résolution mais tous les effets doivent être appliqués et aucun ne peut être annulé à cause de l'ordre de résolution.
\item Un joueur gagne dès qu'il obtient 3 PV et met fin à la partie sans attendre la fin de la ronde.
\end{itemize}



% ### Fin du document ###
\end{multicols}
\end{document}
